\section{Plug-In Responses}
\label{response_section}

The body of the HTTP response of a plug-in consists of a flag indicating the success of the operation, an action, and a data block that is required for processing the action. Responses are delivered in JSON format. The possible actions are listed in the enumeration \textit{PluginResponseType} which is contained in \verb!ServerPlugin.java!. The following sections describe each action in detail.

\subsection{OPEN - Action}
	\begin{table}[ht]
		\centering
		\begin{tabular}{@{}ll@{}}
			\toprule
			\textbf{Description} & Opens a new window or browser tab (depending on browser). \\
			& This can usually be used to open new models. \\
			\addlinespace
			\textbf{Required data} & The URI that should be opened. \\
			\bottomrule						
		\end{tabular}
	\end{table}
	
	\paragraph{Remark} To add new temporary models to the model manager and return the corresponding URL, the code of Listing \ref{newtmp} in Section \ref{code_examples} might help.
	
	\subsubsection{Response Format}
	\begin{lstlisting}
{
	success: true,
	action: 'OPEN',
	uri: '/your/valid/uri/starting/with/slash'
}
	\end{lstlisting}
	
\subsection{UPDATE - Action}
	\begin{table}[ht]
		\centering
		\begin{tabular}{@{}ll@{}}
			\toprule
			\textbf{Description} & Update the current model. This includes changes of node and edge\\
			&representations, their positions, as
			well as their creation, or removal.\\
			&Changes of element positions are animated. \\
			\addlinespace
			\textbf{Required data} & Model difference information. For details see the description in Section \ref{layout_data}\\
			\bottomrule						
		\end{tabular}
	\end{table}
	
	\subsubsection{Model Difference Data}
	\label{layout_data}
	Model difference data is used to communicate server side changes of a model to the editor. It states which nodes and edges have been changed, removed, or created and provides the necessary data. For details on the data format see class \textit{ModelDifferenceTracker}.
	
	Listing \ref{mdt} shows an example of how a \textit{ModelDifferenceTracker} can be used to collect the necessary information. 	
	\subsubsection{Response Format}
		\begin{lstlisting}
{
	success: true,
	action: 'UPDATE',
	data: <model difference data>
}
		\end{lstlisting}	
	
\subsection{INFO - Action}	
	\begin{table}[ht]
		\centering
		\begin{tabular}{@{}ll@{}}
			\toprule
			\textbf{Description} & Display an informative message to the user. \\
			& The generated dialog will show an INFO icon.\\
			\addlinespace
			\textbf{Required data} & The message that should be displayed.\\
			\bottomrule						
		\end{tabular}
	\end{table}
	
	\subsubsection{Response Format}
		\begin{lstlisting}
{
	success: true,
	action: 'INFO',
	infomsg: 'Your message to display'
}
		\end{lstlisting}
	
\subsection{ERROR - Action}
	\begin{table}[ht]
		\centering
		\begin{tabular}{@{}ll@{}}
			\toprule
			\textbf{Description} & Display an error message to the user. \\
			& The generated dialog will show an ERROR icon.\\
			\addlinespace
			\textbf{Required data} & The message that should be displayed.\\
			\bottomrule						
		\end{tabular}
	\end{table}
	
	\subsubsection{JSON Response Format}
		\begin{lstlisting}
{
	success: false
	action: 'ERROR',
	errormsg: 'Your message to display'
}
		\end{lstlisting}	